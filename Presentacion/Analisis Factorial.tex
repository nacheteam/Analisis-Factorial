\documentclass[10pt]{beamer}

\usepackage[utf8]{inputenc}
\usepackage[spanish, es-tabla]{babel}

\usetheme{metropolis}
\usepackage{appendixnumberbeamer}

\usepackage{booktabs}
\usepackage[scale=2]{ccicons}

\usepackage{pgfplots}
\usepgfplotslibrary{dateplot}

\usepackage{caption}
\usepackage{subcaption}

\usepackage{graphicx}

\usepackage{amsmath}
\usepackage{amsfonts}
\usepackage{amssymb}
\usepackage{amsthm}
\usepackage{esvect}

\usepackage{xspace}
\newcommand{\themename}{\textbf{\textsc{metropolis}}\xspace}

\title{Análisis Factorial}
\author{Ignacio Aguilera Martos}
\date{\today}
\institute{Estadística Multivariante \\ \href{https://github.com/nacheteam/Analisis-Factorial}{Link: \underline{Documentos LaTeX en GitHub}}}

\begin{document}

\maketitle

\begin{frame}[fragile]{Contenidos}
  \setbeamertemplate{section in toc}[sections numbered]
  \tableofcontents[hideallsubsections]
\end{frame}

%%%%%%%%%%%%%%%%%%%%%%%%%%%%%%%%%%%%%%%%%%%%%%%%%%%%%%%%%%%%%%%%%%%%%%%%%%%%%%%%%%%%%%%%%%%%%
%%                          Explicación teórica del modelo                                 %%
%%%%%%%%%%%%%%%%%%%%%%%%%%%%%%%%%%%%%%%%%%%%%%%%%%%%%%%%%%%%%%%%%%%%%%%%%%%%%%%%%%%%%%%%%%%%%

\section{Explicación teórica del modelo}

\begin{frame}[fragile]{Modelo}
	\vspace{10px}
	\pause
	\metroset{block=fill}
	\begin{alertblock}{Idea de AF}
		El objetivo de este modelo es, dada una matriz de covarianzas o de correlación, ser capaz de explicar esta matriz a partir de factores no observados llamados factores comunes, de forma que se pueda explicar la matriz con un número menor de variables que en un punto inicial.
	\end{alertblock}
	De esta forma matricialmente si tenemos n observaciones de dimensión p tendríamos una matriz
	$F=
	\begin{pmatrix}
	F_{11} & ... & F_{1n} \\
	... & ... & ... \\
	F_{k1} & ... & F_{kn}
	\end{pmatrix}
	$ de factores y 
	$L=
	\begin{pmatrix}
	l_{11} & ... & l_{1k} \\
	... & ... & ... \\
	l_{p1} & ... & l_{pk}	
	\end{pmatrix}
	$ una matriz de coeficientes de forma que $x-\mu = LF + \epsilon$ donde $\epsilon$ es un vector de errores.
\end{frame}

\begin{frame}[fragile]{Tipos de Análisis Factorial}
\vspace{10px}
\pause
\metroset{block=fill}
\begin{block}{Tipos}
	\begin{itemize}
		\item EFA (Exploratory Factor Analysis): se usa para identificar relaciones complejas entre conceptos o grupos de conceptos.
		\pause
		\item CFA (Confirmatory Factor Analsysis): está dirigida a la confirmación de factores que ya se presuponen importantes para explicar la matriz de correlación o covarianza.
	\end{itemize}
\end{block}
\end{frame}

\begin{frame}[fragile]{EFA}
\vspace{10px}
\pause
\metroset{block=fill}
\begin{block}{Procesos de ajuste}
	\begin{itemize}
		\item Máxima verosimilitud: es una buena opción cuando los datos se distribuyen según una normal. Se intenta que el modelo de los factores obtenidos tenga máxima verosimilitud.
		\pause
		\item Factorización en el eje principal: la intención es ir obteniendo factores de forma que el primero tenga la varianza lo más próxima al objetivo, el segundo factor la segunda varianza más próxima a la varianza objetivo, etc. Maximiza la fórmula del modelo.
	\end{itemize}
\end{block}
\end{frame}

\begin{frame}[fragile]{EFA}
	La rotación de factores se emplea para obtener la estructura de factores más simple escogiendo una orientación de los mismos.
	\vspace{10px}
	\pause
	\metroset{block=fill}
	\begin{block}{Rotación de factores}
		\begin{itemize}
			\item Ortogonal: implica que los factores estén incorrelados y busca la estructura más simple.
			\pause
			\item Oblicua: permite que los factores estén correlados y busca la estructura más simple.
		\end{itemize}
	\end{block}
\end{frame}


\begin{frame}[fragile]{EFA}
No sólo tenemos que estudiar la generación de los factores, si no también cuántos factores debemos escoger.
\vspace{10px}
\pause
\metroset{block=fill}
\begin{block}{Métodos para escoger el número de factores}
	\begin{itemize}
		\item Regla de Kaiser: tomamos los valores propios de la matriz de entrada y comprobamos cuántos de ellos son mayores que 1. Este es el número de factores a tomar. En caso de no haberlo se toma un factor.
		\item Criterio de la gráfica de Cattell's: obtenemos los valores propios de la matriz de entrada y los pintamos de mayor a menor. Analizamos el cambio entre los valores propios y donde se produzca el último cambio brusco contamos el número de valores propios hasta él. Este es el número de factores. Es un método subjetivo y ampliamente criticado.
	\end{itemize}
\end{block}
\end{frame}

\begin{frame}[fragile]{EFA}
\vspace{10px}
\pause
\metroset{block=fill}
\begin{block}{Métodos para escoger el número de factores}
	\begin{itemize}
		\item VSS (Very Simple Structure): este procedimiento toma un modelo simplificado del problema e intenta ver para qué número de factores los valores obtenidos se acercan más a los que deberían ser.
		\item Comparación de modelos: intentan obtener una medida de cómo de bueno y complejo es el modelo creado con un número de factores dado, de forma que se intenta maximizar el resultado y minimizar la complejidad.
		\item OC (Optimal Coordinate): intenta eliminar la subjetividad del método de Cattell. Se calculan los gradientes en la misma gráfica de dicho método y se comprueba dónde hay un cambio más abrupto. Esto delimita el número de factores.
		\item AF (Acceleration Factor): persigue el mismo objetivo que el método anterior, salvo que en este caso se realiza el cálculo con la pendiente de la curva asociada.
	\end{itemize}
\end{block}
\end{frame}

\begin{frame}[fragile]{EFA}
\vspace{10px}
\pause
\metroset{block=fill}
\begin{block}{Métodos para escoger el número de factores}
	\begin{itemize}
		\item MAP (Minimum Average Partial) test: se realiza, desde k=1 hasta el número de variables menos 1 un análisis PCA del modelo con k número de factores. Se estudia cómo se comporta dicho modelo con k factores y se toma el valor de k para el cual se ha obtenido el mejor resultado.
		\item PA (Parallel Analysis): tomamos la misma gráfica que en el método de Cattell y generamos aleatoriamente un conjunto de valores. Estos valores representan los valores medios de los valores propios de matrices aleatorias con el mismo número de variables y datos que la original. Hallamos la media para estos valores y tomamos como número de factores el de los que superen este valor medio. Se puede ver como un refinamiento de la regla de Kaiser ajustando la cota.
	\end{itemize}
\end{block}
\end{frame}

\begin{frame}[fragile]{EFA}
\vspace{10px}
\pause
\metroset{block=fill}
\begin{block}{Métodos para escoger el número de factores}
	\begin{itemize}
		\item Comparación de datos: se realiza una comparación entre modelos con una separación en factores ya conocida como correcta comparando los valores propios de la matriz y los factores escogidos en cada caso. Se toma el número de factores del modelo cuyos valores propios se parezcan más a los del caso que estamos analizando.
		\item Convergencia de múltiples tests: esta estrategia busca de forma empírica el mejor número de factores analizando la convergencia del modelo.
	\end{itemize}
\end{block}
\end{frame}

%%%%%%%%%%%%%%%%%%%%%%%%%%%%%%%%%%%%%%%%%%%%%%%%%%%%%%%%%%%%%%%%%%%%%%%%%%%%%%%%%%%%%%%%%%%%%
%%                         Explicación del ejemplo y uso en R                              %%
%%%%%%%%%%%%%%%%%%%%%%%%%%%%%%%%%%%%%%%%%%%%%%%%%%%%%%%%%%%%%%%%%%%%%%%%%%%%%%%%%%%%%%%%%%%%%

\section{Explicación del ejemplo en R}

\begin{frame}[fragile]{Instalación}
	\vspace{10px}
	\pause
	\metroset{block=fill}
	\begin{block}{Paquetes a instalar}
		\begin{itemize}
			\item install.packages(``psych''): implementa Análisis Factorial
			\pause
			\item install.packages(``GPArotation''): implementa la Rotación de Factores
			\pause
		\end{itemize}
	\end{block}
	\vspace{10px}
	\pause
	\metroset{block=fill}
	\begin{alertblock}{Importa los paquetes}
		\begin{itemize}
			\item library(psych)
			\pause
			\item library(GPArotation)
			\pause
		\end{itemize}
	\end{alertblock}
\end{frame}

\begin{frame}[fragile]{Funciones}
\vspace{10px}
\pause
\metroset{block=fill}
\begin{block}{Funciones a usar}
	\begin{itemize}
		\item fa: función del paquete psych que implementa Análisis Factorial.
		\pause
		\item factanal: función del paquete stats que implementa Análisis Factorial Exploratorio con máxima verosimilitud.
		\pause
		\item prcomp: función del paquete stats que implementa PCA.
		\pause
		\item principal: función del paquete psych que implementa PCA devolviendo siempre los mejores factores.
		\item cfa: función del paquete cfa que implementa el análisis confirmatorio de factores.
		\pause
	\end{itemize}
	En este caso he usado la función fa como función principal y factanal. Para comparar con PCA he usado principal. Para comprar con CFA he usado cfa.
\end{block}
\end{frame}

\begin{frame}[standout]
	\LARGE{¿Preguntas?}
	\vspace{10px}
	\begin{figure}
		\includegraphics[scale=0.5]{./Imagenes/preguntas.png}
	\end{figure}
\end{frame}


\end{document}
