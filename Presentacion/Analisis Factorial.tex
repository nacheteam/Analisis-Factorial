\documentclass[10pt]{beamer}

\usepackage[utf8]{inputenc}
\usepackage[spanish, es-tabla]{babel}

\usetheme{metropolis}
\usepackage{appendixnumberbeamer}

\usepackage{booktabs}
\usepackage[scale=2]{ccicons}

\usepackage{pgfplots}
\usepgfplotslibrary{dateplot}

\usepackage{caption}
\usepackage{subcaption}

\usepackage{graphicx}

\usepackage{amsmath}
\usepackage{amsfonts}
\usepackage{amssymb}
\usepackage{amsthm}
\usepackage{esvect}

\usepackage{xspace}
\newcommand{\themename}{\textbf{\textsc{metropolis}}\xspace}

\title{Análisis Factorial}
\author{Ignacio Aguilera Martos}
\date{\today}
\institute{Estadística Multivariante \\ \href{https://github.com/nacheteam/Analisis-Factorial}{Link: \underline{Documentos LaTeX en GitHub}}}

\begin{document}

\maketitle

\begin{frame}[fragile]{Contenidos}
  \setbeamertemplate{section in toc}[sections numbered]
  \tableofcontents[hideallsubsections]
\end{frame}

\begin{frame}[fragile]{Instalación}
	\vspace{10px}
	\pause
	\metroset{block=fill}
	\begin{block}{Paquetes a instalar}
		\begin{itemize}
			\item install.packages(``psych''): implementa Análisis Factorial
			\pause
			\item install.packages(``GPArotation''): implementa la Rotación de Factores
			\pause
		\end{itemize}
	\end{block}
	\vspace{10px}
	\pause
	\metroset{block=fill}
	\begin{alertblock}{Importa los paquetes}
		\begin{itemize}
			\item library(psych)
			\pause
			\item library(GPArotation)
			\pause
		\end{itemize}
	\end{alertblock}
\end{frame}

\begin{frame}[fragile]{Función fa}
\vspace{10px}
\pause
\metroset{block=fill}
\begin{block}{Funciones a usar}
	\begin{itemize}
		\item fa: función del paquete psych que implementa Análisis Factorial.
		\pause
		\item factanal: función del paquete stats que implementa Análisis Factorial Exploratorio con máxima verosimilitud.
		\pause
		\item prcomp: función del paquete stats que implementa PCA.
		\pause
		\item principal: función del paquete psych que implementa PCA devolviendo siempre los mejores factores.
	\end{itemize}
\end{block}
\end{frame}

\begin{frame}[standout]
	\LARGE{¿Preguntas?}
	\vspace{10px}
	\begin{figure}
		\includegraphics[scale=0.5]{./Imagenes/preguntas.png}
	\end{figure}
\end{frame}


\end{document}
